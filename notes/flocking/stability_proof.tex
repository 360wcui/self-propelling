%% LyX 2.3.4.2 created this file.  For more info, see http://www.lyx.org/.
%% Do not edit unless you really know what you are doing.
\documentclass[english]{article}
\usepackage[T1]{fontenc}
\usepackage[latin9]{inputenc}
\usepackage{geometry}
\geometry{verbose,tmargin=1in,bmargin=1in,lmargin=1in,rmargin=1in}
\usepackage{verbatim}
\usepackage{amsmath}
\usepackage{amssymb}

\makeatletter
%%%%%%%%%%%%%%%%%%%%%%%%%%%%%% User specified LaTeX commands.
\newcommand{\RR}{\mathbb{R}^{2}}
\newcommand{\Qi}{Q_{i}}
\newcommand{\Qj}{Q_{j}}
\newcommand{\sgnDist}{b}
\newcommand{\posi}{q_{i}}
\newcommand{\pos}{q}
\newcommand{\posrel}{x}
\newcommand{\vel}{p}
\newcommand{\velrel}{v}
\newcommand{\pix}{q_{i,x}}
\newcommand{\piy}{q_{i,y}}
\newcommand{\veli}{p_{i}}
\newcommand{\smax}{s_{max}}
\newcommand{\vix}{p_{i,x}}
\newcommand{\viy}{p_{i,y}}
\newcommand{\ui}{u_i}
\newcommand{\uix}{u_{i,x}}
\newcommand{\uiy}{u_{i,y}}
\newcommand{\uis}{u_{i,s}}
\newcommand{\uitheta}{u_{i,\theta}}
\newcommand{\pjx}{q_{j,x}}
\newcommand{\pjy}{q_{j,y}}
\newcommand{\vjx}{v_{j,x}}
\newcommand{\vjy}{v_{j,y}}
\newcommand{\ujx}{u_{j,x}}
\newcommand{\ujy}{u_{j,y}}
\newcommand{\dx}{d_{x}}
\newcommand{\dy}{d_{y}}
\newcommand{\thetai}{\theta_{i}}
\newcommand{\si}{s_{i}}

\newcommand{\vxr}{v_{r,x}}
\newcommand{\vyr}{v_{r,y}}
\newcommand{\pxr}{p_{r,x}}
\newcommand{\pyr}{p_{r,y}}
\newcommand{\vdxr}{\dot{v}_{r,x}}
\newcommand{\vdyr}{\dot{v}_{r,y}}
\newcommand{\pdxr}{\dot{p}_{r,x}}
\newcommand{\pdyr}{\dot{p}_{r,y}}

\newcommand{\ttr}{\psi}
\newcommand{\rd}{r_d}
\newcommand{\rO}{r_0}
\newcommand{\hO}{h_0}

\makeatother

\usepackage{babel}
\begin{document}

\section{Methodology - Coverage Controller}

We consider a group of $N$ vehicles, each of them denoted $Q_{i}$,
$i=1,\cdots,N$, with dynamics described by 
\[
\dot{\posi}=\veli,\quad\dot{\veli}=\ui.
\]

Let $\Omega\subseteq\mathbb{R}^{2}$ be a compact domain containing
zero, and define $\Omega\left(t\right)=\Omega+\pos_{d}\left(t\right)$,
where $\pos_{d}\left(t\right)$ is the solution for the system
\[
\begin{cases}
\dot{\pos_{d}} & =\vel_{d}\\
\dot{\vel_{d}} & =f_{d}\left(\pos_{d},\vel_{d}\right),
\end{cases}
\]

\noindent we call $\pos_{d}\left(t\right)$ the marker point of the
moving domain $\Omega\left(t\right)$. 

Define $\pos_{ij}:=\pos_{i}-\pos_{j}$, and $\vel_{ij}:=\vel_{i}-\vel_{j}$
and denote by $P_{\partial\Omega\left(t\right)}\left(\pos_{i}\right)$
the closest point of $\partial\Omega\left(t\right)$ to $\pos_{i}$
(i.e., the projection of $\pos_{i}$ on $\partial\Omega\left(t\right)$).
Also, define $h_{i}:=\pos_{i}-P_{\partial\Omega\left(t\right)}\left(\pos_{i}\right)$,
and denote by $\left[\left[h_{i}\right]\right]$ the signed distance
of $\pos_{i}$ from $\partial\Omega\left(t\right)$.

The proposed control force is given by:

\begin{equation}
u_{i}=-\sum_{j\neq i}^{N}f_{I}\left(\left\Vert \pos_{ij}\right\Vert \right)\frac{\pos_{ij}}{\left\Vert \pos_{ij}\right\Vert }-\frac{1}{N}\sum_{j\neq i}^{N}f_{al}\left(\left\Vert \pos_{ij}\right\Vert \right)\vel_{ij}-f_{h}\left(\left[\left[h_{i}\right]\right]\right)\frac{h_{i}}{\left[\left[h_{i}\right]\right]}-a\left(\vel_{i}-\vel_{d}\right)\label{eq:Coverage-controller}
\end{equation}

\begin{comment}
As done in {[}ref{]} we choose $f_{al}\left(\left\Vert \pos_{ij}\right\Vert \right)=C_{al}e^{-\frac{\left\Vert \pos_{ij}\right\Vert }{l_{al}}}$,
where $C_{al}$ and $l_{al}$ are constants associated to the velocity
alignment strength and 
\end{comment}

The position and velocity of the $i$ th vehicle relative to the marker
of the moving domain are given by: 
\[
\begin{cases}
\posrel_{i} & :=\pos_{i}-\pos_{d}\\
\velrel_{i} & :=\vel_{i}-\vel_{d}.
\end{cases}
\]

Note that the inter-vehicle position and velocity in this new framework
satisfy:
\begin{align*}
\posrel_{ij} & :=\posrel_{i}-\posrel_{j}=\pos_{i}-\pos_{d}-\left(\pos_{j}-\pos_{d}\right)=\pos_{ij},\\
\velrel_{ij} & :=\velrel_{i}-\velrel_{j}=\vel_{i}-\vel_{d}-\left(\vel_{j}-\vel_{d}\right)=\vel_{ij},
\end{align*}

\noindent it means the relative positions are invariant to the change
of coordinates. Moreover, the vehicle domain distance satisfies $h_{i}=\pos_{i}-P_{\partial\Omega\left(t\right)}\left(\pos_{i}\right)=\left(\pos_{i}-\pos_{d}\right)-P_{\partial\Omega\left(t\right)-\pos_{d}}\left(\pos_{i}-\pos_{d}\right)=\posrel_{i}-P_{\partial\Omega\left(0\right)}\left(\posrel_{i}\right)$.
This allow us to rewrite (\ref{eq:Coverage-controller}) as
\begin{equation}
u_{i}=-\sum_{j\neq i}^{N}f_{I}\left(\left\Vert \posrel_{ij}\right\Vert \right)\frac{\posrel_{ij}}{\left\Vert \posrel_{ij}\right\Vert }-\frac{1}{N}\sum_{j\neq i}^{N}f_{al}\left(\left\Vert \posrel_{ij}\right\Vert \right)\velrel_{ij}-f_{h}\left(\left[\left[h_{i}\right]\right]\right)\frac{h_{i}}{\left[\left[h_{i}\right]\right]}-a\velrel_{i}\label{eq:Coverage-controller-rel}
\end{equation}

Let us consider the potential 
\[
V_{h}\left(\posrel_{i}\right)=\int_{-\frac{r_{d}}{2}}^{\left[\left[\posrel_{i}-P_{\partial\Omega\left(0\right)}\left(\posrel_{i}\right)\right]\right]}f_{h}\left(s\right)ds
\]

\noindent which satisfies
\[
\nabla_{\posrel_{i}}V_{h}\left(\posrel_{i}\right)=f_{h}\left(\left[\left[\posrel_{i}-P_{\partial\Omega\left(0\right)}\left(\posrel_{i}\right)\right]\right]\right)\nabla_{\posrel_{i}}\left(\left[\left[\posrel_{i}-P_{\partial\Omega\left(0\right)}\left(\posrel_{i}\right)\right]\right]\right)=f_{h}\left(\left[\left[h_{i}\right]\right]\right)\frac{h_{i}}{\left[\left[h_{i}\right]\right]}
\]

\noindent where we have used the identity $\nabla_{\posrel_{i}}\left(\left[\left[\posrel_{i}-P_{\partial\Omega\left(0\right)}\left(\posrel_{i}\right)\right]\right]\right)=\frac{\posrel_{i}-P_{\partial\Omega\left(0\right)}\left(\posrel_{i}\right)}{\left[\left[\posrel_{i}-P_{\partial\Omega\left(0\right)}\left(\posrel_{i}\right)\right]\right]}$.

Similarly, it can be shown that the inter-vehicle force is the negative
gradient of the potential 
\[
V_{I}\left(\posrel_{ij}\right)=\int_{r_{d}}^{\left\Vert \posrel_{ij}\right\Vert }f_{I}\left(s\right)ds,
\]

\noindent to finally get:
\begin{equation}
u_{i}=\underset{\text{Inter Vehicle}}{\underbrace{-\sum_{j\neq i}^{N}\nabla_{\posrel_{i}}V_{I}\left(\posrel_{ij}\right)}}-\underset{\text{Velocity Alignment}}{\underbrace{\frac{1}{N}\sum_{j\neq i}^{N}f_{al}\left(\left\Vert \posrel_{ij}\right\Vert \right)\velrel_{ij}}}-\overset{\text{Navigational feedback}}{\overbrace{\underset{\text{Domain Vehicle}}{\underbrace{\nabla_{\posrel_{i}}V_{h}\left(\posrel_{i}\right)}}-\underset{\text{Speed Alignment}}{\underbrace{a\velrel_{i}}}}}\label{eq:Coverage-controller-rel-1}
\end{equation}

\noindent Consider the candidate for Lyapunov function consisting
in kinetic plus (artificial) potential energy:
\[
\Phi=\frac{1}{2}\underset{i=1}{\overset{N}{\sum}}\Bigl(\dot{\posrel_{i}}\cdot\dot{\posrel_{i}}+\underset{j\neq i}{\overset{N}{\sum}}V_{I}\left(\posrel_{ij}\right)+V_{h}\left(\posrel_{i}\right)\Bigr).
\]

\noindent Note that each term in $\Phi$ is non-negative, and $\Phi$
reaches its absolute minimum value when the vehicles are totally stopped. 

The derivative of $\Phi$ with respect to time can be calculated as:
\begin{align*}
\dot{\Phi} & =\underset{i=1}{\overset{N}{\sum}}\dot{\posrel_{i}}\cdot\Bigl(u_{i}+\underset{j\neq i}{\overset{N}{\sum}}\nabla_{\posrel_{i}}V_{I}\left(\posrel_{ij}\right)+\nabla_{\posrel_{i}}V_{h}\left(\posrel_{i}\right)\Bigr)\\
 & =\underset{i=1}{\overset{N}{\sum}}\dot{\posrel_{i}}\cdot\left(-\frac{1}{N}\sum_{j\neq i}^{N}f_{al}\left(\left\Vert \posrel_{ij}\right\Vert \right)\velrel_{ij}-a\velrel_{i}\right)
\end{align*}

\noindent For the (extra) alignment term, write 
\[
\underset{i=1}{\overset{N}{\sum}}\velrel_{i}\cdot\underset{j\neq i}{\overset{N}{\sum}}f_{al}\left(\left\Vert \posrel_{ij}\right\Vert \right)\left(\velrel_{i}-\velrel_{j}\right)=\frac{1}{2}\underset{i=1}{\overset{N}{\sum}}\velrel_{i}\cdot\underset{j\neq i}{\overset{N}{\sum}}f_{al}\left(\left\Vert \posrel_{ij}\right\Vert \right)\left(\velrel_{i}-\velrel_{j}\right)+\frac{1}{2}\underset{j=1}{\overset{N}{\sum}}\velrel_{j}\cdot\underset{i\neq j}{\overset{N}{\sum}}f_{al}\left(\left\Vert \posrel_{ij}\right\Vert \right)\left(\velrel_{j}-\velrel_{i}\right),
\]

\noindent where in the second term in the right-hand-side we simply
rename $i\leftrightarrow j$ as indices of summation. From there,
use that $\|\posrel_{ij}\|=\|\posrel_{ji}\|$ to get:
\[
\underset{i=1}{\overset{N}{\sum}}\velrel_{i}\cdot\underset{j\neq i}{\overset{N}{\sum}}f_{al}\left(\left\Vert \posrel_{ij}\right\Vert \right)\left(\velrel_{i}-\velrel_{j}\right)=\frac{1}{2}\underset{i=1}{\overset{N}{\sum}}\underset{j\neq i}{\overset{N}{\sum}}f_{al}\left(\left\Vert \posrel_{ij}\right\Vert \right)\|\velrel_{i}-\velrel_{j}\|^{2}
\]
.

\noindent With the minus sign in front this gives a negative-definite
term. Conclude that $\dot{\Phi}$ is negative semidefinite and equal
to zero if and only if $\dot{\posrel_{i}}=0$ for all $i$ (i.e.,
all vehicles are at equilibrium in the relative framework). 
\end{document}
